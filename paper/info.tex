\documentclass{article}
\usepackage[utf8]{inputenc}
\usepackage[cm]{fullpage}

\usepackage{booktabs}
\usepackage{amsmath, amssymb, amsthm, hyperref}
\usepackage{multicol}
\usepackage{tablefootnote}
\usepackage{xspace}
\usepackage[shortlabels, inline]{enumitem}

\usepackage{hyperref}
\hypersetup{
  colorlinks=true,
  citecolor=black,
  filecolor=black,
  linkcolor=blue,
  urlcolor=red
}
\usepackage[capitalise, nameinlink]{cleveref}

\theoremstyle{definition}
\newtheorem{defn}{Definition}[section]
\newtheorem{remark}{Remark}

\theoremstyle{plain}
\newtheorem{cor}[defn]{Corollary}
\newtheorem{theorem}[defn]{Theorem}

\setlength{\unitlength}{0.5mm}
\renewcommand{\labelenumi}{{\normalfont(\alph{enumi})}}
\setenumerate{noitemsep, topsep=0pt}

%%%%%%%%%%%%%%%%%%%%%%%%%%%%%%%%%%%%%%%%%%%%%%%%%%%%%%%%%%%%%%%%%%%%%%%%
% Bibliography
%%%%%%%%%%%%%%%%%%%%%%%%%%%%%%%%%%%%%%%%%%%%%%%%%%%%%%%%%%%%%%%%%%%%%%%%

\usepackage[bibencoding=utf8, giveninits=true, sortcites]{biblatex}
\addbibresource{info.bib}

%%%%%%%%%%%%%%%%%%%%%%%%%%%%%%%%%%%%%%%%%%%%%%%%%%%%%%%%%%%%%%%%%%%%%%%%
% User-defined macros
%%%%%%%%%%%%%%%%%%%%%%%%%%%%%%%%%%%%%%%%%%%%%%%%%%%%%%%%%%%%%%%%%%%%%%%%

\newcommand{\GAP}{\textsc{GAP}~\cite{GAP4}\xspace}
\newcommand{\Smallsemi}{\textsc{Smallsemi}~\cite{Smallsemi}\xspace}
\newcommand{\Semigroups}{\textsc{Semigroups}~\cite{Semigroups}\xspace}
\newcommand{\Sym}{\operatorname{Sym}}
\newcommand{\Aut}{\operatorname{Aut}}

%%%%%%%%%%%%%%%%%%%%%%%%%%%%%%%%%%%%%%%%%%%%%%%%%%%%%%%%%%%%%%%%%%%%%%%%
% Preamble ends
%%%%%%%%%%%%%%%%%%%%%%%%%%%%%%%%%%%%%%%%%%%%%%%%%%%%%%%%%%%%%%%%%%%%%%%%

\title{Enumerating finite semirings}
\author{J. D. Mitchell and P. Ragavan}
\date{\today}

\begin{document}

\maketitle

\begin{abstract}
  In this short note we count the finite semirings up to isomorphism
  and up to anti-isomorphism for some small values of $n$; for which we
  utilise the existing library of small semigroups in the \GAP
  package \Smallsemi.
\end{abstract}

\section{Introduction}

% TODO(JDM): fill in the other non-OEIS references

Enumeration of algebra and combinatorial structures of finite order up to
isomorphism is a classical topic. Among the algebraic structures considered are
groups~\cite{BESCHE2002,}, rings~\cite{Blackburn2022, Fine1993, Kruse1970,
A027623}, near-rings~\cite{SONATA, A305858}, semigroups~\cite{A027851},
monoids~\cite{A058129}, inverse semigroups~\cite{A001428},
loops~\cite{A000315}, quasi-groups~\cite{A002860}, functional
digraphs~\cite{A001373} (or mono-unary algebras to some) and many more too
numerous to mention. In this short note we count the number of finite semirings
up to isomorphism and up to isomorphism and anti-isomorphism for $n \leq 6$. We
also count several special classes of semirings for (slightly) larger values of
$n$.

This short note was initiated by an email from M. Volkov to the first author in
February of 2025 asking if it was possible to verify with \GAP that the number
of ai-semirings up to isomorphism with $4$ elements is $866$.
% TODO(JDM) Possibly refer to the paper that Misha sent me at the time?
After some initial missteps it was relatively straightforward to verify that
this number is correct, by using the library of small semigroups in the \GAP
package \Smallsemi. This short note arose out from these first steps. In
contrast to groups or rings, where the numbers of non-isomorphic objects of
order $n$ is known for relatively large values of $n$, the number of semigroups
of order $11$ (up to isomorphism) is not known exactly. Given that $99.4\%$ of
the semigroups of order $8$ are $3$-nilpotent, that the number of $3$-nilpotent
semigroups of order $11$ is approximately $10 ^ {26}$~\cite{}, this number is
likely close to the exact value; see also~\cite{}. Perhaps unsurprisingly, from
the perspective of counting up to isomorphism, it seems that semirings have
more in common with semigroups than with rings or groups. Roughly speaking,
rings and groups are highly structured, providing strong constraints that
enable their enumeration. On the other hand, semigroups, and seemingly
semirings also, are less structured, more numerous, and consequently harder to
enumerate.

We begin with the definition of a semiring; this originates in
\cite{} and was intended as a generalisation of the notion of a ring.

\begin{defn}[Semiring]
  \label{def:semiring}
  A \emph{semiring} is a set \(S\) equipped with two binary
  operations \(+\) and  \(\times\) such that:
  \begin{enumerate}
    \item \((S, +)\) is a commutative semigroup ($(x + y) + z = x + (y + z)$
      and $x + y = y + x$ for all $x, y, z\in S$);
    \item \((S, \times)\) is a semigroup ($x\times (y \times z) = (x\times
      y)\times z$ for all $x,y, z\in S$); and
    \item multiplication $\times$ distributes over addition $+$
      ($x \times (y + z) = (x \times y) + (x \times z)$ and
      $(y + z) \times x = (y \times x) + (z \times x)$ for all $x, y, z \in S$).
  \end{enumerate}
\end{defn}

In this short note we are concerned with counting semirings up to isomorphism,
and so our next definition is that of an isomorphism.

% TODO mappings written on the right and composed left to right.

\begin{defn}[Semiring isomorphism]\label{defn-semiring-iso}
  We say that two semirings \((S,+,\times)\) and \((S, \oplus, \otimes)\) are
  \emph{isomorphic} if there exists a bijection \(\phi: S \to S\) which is
  simultaneously a semigroup isomorphism from $(S, +)$ to $(S, \oplus)$ and
  from $(S, \times)$ to $(S, \otimes)$. We refer to $\phi$ as a
  \textit{(semiring) isomorphism}.
\end{defn}

If \((+,\times)=(\oplus, \otimes)\) in \cref{defn-semiring-iso}, then the
semiring isomorphism $\phi$ is called an \textit{automorphism}. The group of
all automorphisms of a semiring $S$ is denoted by $\Aut(S)$.
% TODO need semigroup automorphisms not semiring
Since a semiring is comprised of two semigroups, enumerating semirings is
equivalent to enumerating those pairs consisting of an additive semigroup $(S,
+)$ and a multiplicative semigroup $(S, \times)$ such that $\times$ distributes
over $+$. The next theorem indicates which $(S, \times)$ we should consider
for each of the additive semigroups $(S, +)$.

We denote the symmetric group on the set $S$ by $\Sym(S)$.
If $\sigma\in \Sym(S)$ and
$\cdot: S \times S \to S$ is a binary operation, then we define the binary
operation $\cdot ^ \sigma: S\times S \to S$ by
\begin{equation}\label{equation-action}
  x \cdot ^ \sigma y = ((x)\sigma^{-1} \cdot (y)\sigma ^ {-1})\sigma.
\end{equation}
It is straightforward to verify that \eqref{equation-action} is a (right group)
action of $\Sym(S)$ on the set of all binary operations on $S$. Clearly if
$\cdot$ is associative, then so too is $\cdot ^ \sigma$ for every $\sigma \in
\Sym(S)$. The group of automorphisms $\Aut(S,\cdot)$ of a
semigroup $(S, \cdot)$ coincides with the stabiliser of the operation $\cdot$
under the action of $\Sym(S)$ defined in \eqref{equation-action}.

Recall that if $H$ and $K$ are subgroups of a group $G$, then the
\textit{double cosets} $H\backslash G / K$ are the sets of the form
$\{hgk\mid h\in H, k \in K\}$ for $g\in G$.
The next theorem is key to our approach for counting semirings.

\begin{theorem}
  \label{thm:isomorphism-condition}
  Let \((S, +)\) be a commutative semigroup, let \((S, \times)\) be a semigroup,
  and let \(\sigma, \tau\in\Sym(S)\) be such that $(S, +, \times ^ \sigma)$ and
  $(S, +, \times ^ \tau)$ are semirings.
  Then \((S, +, \times ^ \sigma)\) and \((S, +, \times^\tau)\) are isomorphic
  if and only if \(\sigma\) and \(\tau\) belong to the same double coset of
  \(\Aut(S, \times) \backslash \Sym(S) / \Aut(S, +)\).
\end{theorem}

\begin{proof}
  ($\Rightarrow$) Suppose that $\phi$ is a semiring isomorphism from
  $(S, +, \times ^ \sigma)$ to $(S, +, \times^\tau)$. Then
  $\phi \in \Aut(S, +)$ and $(x \times ^ \sigma y)\phi = (x)\phi\times ^ \tau
  (y)\phi$ for all $x, y\in S$. It follows that
  \[
    ((x)\sigma ^ {-1} \times (y)\sigma ^ {-1})\sigma \phi \tau ^ {-1} =
    (x \times ^ \sigma y)\phi\tau ^ {-1}
    = ((x)\phi \times ^ {\tau} (y)\phi)\tau ^ {-1}
    = (x)\phi\tau ^ {-1} \times (y)\phi\tau ^ {-1}.
  \]
  If we set $\gamma = \sigma\phi\tau^{-1}$, $p =
  (x)\sigma^{-1}$, and $q=(y)\sigma^{-1}$, then
  $(p\times q)\gamma = (p)\gamma \times (q)\gamma$
  and so $\gamma, \gamma ^ {-1} \in \Aut(S, \times)$.
  Rearranging we obtain $\tau = \gamma^{-1}\sigma\phi$. Since $\phi \in
  \Aut(S, +)$, we conclude that $\tau$ and $\sigma$ lie in the
  same double coset of $\Aut(S, \times) \backslash \Sym(S) / \Aut(S, +)$.

  ($\Leftarrow$) Suppose that $\sigma$ and $\tau$ are in the
  same double coset of $\Aut(S, +) \backslash \Sym(S) / \Aut(S, \times)$.
  Then there exists $\alpha\in\Aut(S, \times)$ and $\beta\in\Aut(S, +)$ such
  that $\tau=\alpha\sigma\beta$.

  We will show that $\beta$ is a semiring isomorphism from $(S, +, \times
  ^{\sigma})$ to $(S, +, \times^{\tau})$. Since $\beta\in \Aut(S, +)$, it
  suffices to show that $\beta$ is an isomorphism from $(S, \times
  ^{\sigma})$ and $(S, \times^{\tau})$:
  \begin{align*}
    (x\times^\sigma y)\beta
    &= (x\sigma ^{-1} \times y\sigma ^{-1})\sigma \beta
    = (x\sigma ^{-1} \times y\sigma ^{-1})\alpha^{-1}\tau
    = (x\sigma ^{-1}\alpha^{-1} \times y\sigma ^{-1}\alpha^{-1})\tau &&
    \alpha^{-1} \in \Aut(S, \times)\\
    &= (x\beta\tau^{-1} \times y\beta\tau^{-1})\tau
    = (x\beta \times^{\tau} y\beta).\qedhere
  \end{align*}
\end{proof}

If $(S, \times)$ and $(S, \otimes)$ are semigroups, then $(S, \times)$ is said
to be \textit{anti-isomorphic} to $(S, \otimes)$ if there exists a bijection
$\phi: S \to S$ such that $(x\times y)\phi = y\otimes x$ for all $x, y\in S$.
The bijection $\phi$ is referred to as an \textit{anti-isomorphism}. Similarly,
using the obvious analogue of \cref{defn-semiring-iso}, we can define
anti-isomorphic semigroups. It is routine to adapt the proof of
\cref{thm:isomorphism-condition} to prove the following.

\begin{cor}
  \label{cor:equiv-condition}
  Let \((S, +)\) be a commutative semigroup, let \((S, \times)\) be a semigroup,
  and let \(\sigma, \tau\in\Sym(S)\) be such that $(S, +, \times ^ \sigma)$ and
  $(S, +, \times ^ \tau)$ are semirings.
  Then \((S, +, \times ^ \sigma)\) and \((S, +, \times^\tau)\) are isomorphic or anti-isomorphic if and only if \(\sigma\) and \(\tau\) belong to the same double coset of \(\Aut^*(S, \times) \backslash \Sym(S) / \Aut(S, +)\).
  where \(\Aut^*(S, \times)\) denotes the group of
  automorphisms and anti-automorphisms of \((S, \times)\).
\end{cor}

\section{Tables of results}

It follows from \cref{thm:isomorphism-condition} and
\cref{cor:equiv-condition} that to enumerate the
semirings on a set $S$ up to isomorphism (or up to isomorphism and
anti-isomorphism) it suffices to consider every
commutative semigroup $(S, +)$ and semigroup $(S, \times)$
and to compute representatives of the double cosets \(\Aut(S, \times)
\backslash \Sym(S) / \Aut(S, +)\). The semigroups up to isomorphism are
available in \Smallsemi.  Since $S$ is small, it is relatively
straightforward to can compute $\Aut(S, \times)$ and $\Aut(S, +)$ (using the
\Semigroups package for \GAP).
\GAP contains functionality for computing double coset
representatives based on \cite{}. This is the approach implemented by the
authors of the current paper in the \GAP package \texttt{semirings} to compute
the numbers in this section.
% TODO(PR) proper reference for semirings repo

% TODO(PR): check which of the numbers in this section might be known already
% (i.e. check the oeis) + add a proper reference to Jipsen
Note that in a structure that does not have additive inverses but
does have a zero element, we typically require the additional axiom
\begin{equation}
  \label{eq:rig-axiom}
  0\cdot a=a\cdot0=0\qquad\forall a\in S,
\end{equation}

Some structures that can be counted using the
\texttt{semirings} package are semirings, semirings with one, ai-semirings, ai-semirings with one and zero, semirings with one and zero, ai-semirings with one and zero, semirings with zero, ai-semirings with zero, rings, and rings with one. By the prefix `ai-' we mean additively idempotent, i.e. $a + a = a$ for all $a\in S$.

Below are some tables of results for the aforementioned structures.
As far as we know, no results are published the number of any of
these structures up to equivalence. For results up to isomorphism,
those that have not been previously published (as far as we know) are
marked `$^\dagger$'. As a sanity check, various results that are already
published are available at Jipsen's Mathematical
Structures Library \cite{MathStructures}, though he may make use of different naming
conventions\footnote{Note that Jipsen's page
  for \emph{semirings with one} \cite{MSsemiringsWithOne}, seems to be mistitled and actually
  provides results for ai-semirings with one (which can be counted
  using the \texttt{semirings} package). This is not a difference in
  naming convention, but seems to just be a mistake. As far as we know,
all results in Table~\ref{tab:unital-semirings} are unpublished.}.
\begin{table}[ht]
  \centering
  \begin{tabular}{l|r|r}
    \toprule
    $n$ & up to isomorphism & up to isomorphism + anti-isomorphism \\
    \midrule
    1 & 1         & 1         \\
    2 & 10        & 9         \\
    3 & 132       & 106       \\
    4 & 2,341     & 1,713     \\
    5 & 57,427$^\dagger$    & 38,247    \\
    6 & 7,571,579$^\dagger$  & 4,102,358 \\
  \end{tabular}
  \caption{Numbers of semirings with $n$ elements up to isomorphism and up
    to isomorphism or anti-isomorphism. See \cite{MSsemirings} for \(n\leq4\) up to isomorphism.}
  \label{tab:semirings}
\end{table}

\begin{table}[ht]
  \centering
  \begin{tabular}{l|r|r}
    \toprule
    $n$ & up to isomorphism & up to isomorphism + anti-isomorphism \\
    \midrule
    1 & 1         & 1       \\
    2 & 6         & 5       \\
    3 & 61        & 45      \\
    4 & 866       & 581     \\
    5 & 15,751$^\dagger$    & 9,750   \\
    6 & 354,409$^\dagger$   & 205,744 \\
    7 & 9,908,909$^\dagger$ & 5,470,437       \\
  \end{tabular}
  \caption{Numbers of ai-semirings with $n$ elements up to isomorphism and up
    to isomorphism or anti-isomorphism. See \cite{MSidempotentSemirings} for \(n\leq4\) up to isomorphism.}
  \label{tab:ai-semirings}
\end{table}

\begin{table}[ht]
  \centering
  \begin{tabular}{l|r|r}
    \toprule
    $n$ & up to isomorphism & up to isomorphism + anti-isomorphism \\
    \midrule
    1 & 1         & 1      \\
    2 & 2         & 2      \\
    3 & 6         & 6      \\
    4 & 40        & 38     \\
    5 & 295       & 262    \\
    6 & 3,246     & 2,681  \\
    7 & 59,314$^\dagger$    & 43,331 \\
  \end{tabular}
  \caption{Numbers of semirings with one and zero with $n$ elements
    up to isomorphism and up
    to isomorphism or anti-isomorphism. See \cite{MSsemiringsWithOneAndZero} for \(n\leq6\) up to isomorphism.}
  \label{tab:semirings-with-one-and-zero}
\end{table}

\begin{table}[ht]
  \centering
  \begin{tabular}{l|r|r}
    \toprule
    $n$ & up to isomorphism & up to isomorphism + anti-isomorphism \\
    \midrule
    1 & 1         & 1      \\
    2 & 1         & 1      \\
    3 & 3         & 3      \\
    4 & 20        & 18     \\
    5 & 149       & 125    \\
    6 & 1,488     & 1,150  \\
    7 & 18,554    & 13,171 \\
    8 & 295,292$^\dagger$   & 116,274  \\
  \end{tabular}
  \caption{Numbers of ai-semirings with one and zero with $n$
    elements up to isomorphism and up
    to isomorphism or anti-isomorphism. See \cite{MSAiSemiringsWithOneAndZero} for \(n\leq7\) up to isomorphism.}
  \label{tab:ai-semirings-with-one-and-zero}
\end{table}

\begin{table}[h]
  \centering
  \begin{tabular}{l|r|r}
    \toprule
    $n$ & up to isomorphism & up to isomorphism + anti-isomorphism \\
    \midrule
    1 & 1         & 1      \\
    2 & 4         & 4      \\
    3 & 22        & 21     \\
    4 & 169       & 155    \\
    5 & 1,819     & 1,561  \\
    6 & 41,104    & 30,112 \\
  \end{tabular}
  \caption{Numbers of semirings with one (unital semirings) with $n$
  elements up to isomorphism and up to isomorphism or anti-isomorphism.}
  \label{tab:unital-semirings}
\end{table}

These tables are merely a sample of the results that can be obtained
using the \texttt{semirings} package.

%TODO(PR) add as many tables as possible that contain numbers that weren't
%known before.
% * semifields
% * zero sum free
% * commutative (with/without 0 and 1)

\printbibliography

\end{document}
