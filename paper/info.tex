\documentclass{article}
\usepackage[utf8]{inputenc}
\usepackage[cm]{fullpage}

\usepackage{booktabs}
\usepackage{amsmath, amssymb, amsthm, hyperref}
\usepackage{multicol}
\usepackage{multirow}
\usepackage{tablefootnote}
\usepackage{xspace}
\usepackage[shortlabels, inline]{enumitem}

\usepackage{hyperref}
\hypersetup{
  colorlinks=true,
  citecolor=black,
  filecolor=black,
  linkcolor=blue,
  urlcolor=red
}
\usepackage[capitalise, nameinlink]{cleveref}

\theoremstyle{definition}
\newtheorem{defn}{Definition}[section]
\newtheorem{remark}{Remark}

\theoremstyle{plain}
\newtheorem{cor}[defn]{Corollary}
\newtheorem{theorem}[defn]{Theorem}

\setlength{\unitlength}{0.5mm}
\renewcommand{\labelenumi}{{\normalfont(\alph{enumi})}}
\setenumerate{noitemsep, topsep=0pt}

% \DeclareUnicodeCharacter{202F}{XXX}

%%%%%%%%%%%%%%%%%%%%%%%%%%%%%%%%%%%%%%%%%%%%%%%%%%%%%%%%%%%%%%%%%%%%%%%%
% Bibliography
%%%%%%%%%%%%%%%%%%%%%%%%%%%%%%%%%%%%%%%%%%%%%%%%%%%%%%%%%%%%%%%%%%%%%%%%

\usepackage[bibencoding=utf8, giveninits=true, sortcites]{biblatex}
\addbibresource{info.bib}

%%%%%%%%%%%%%%%%%%%%%%%%%%%%%%%%%%%%%%%%%%%%%%%%%%%%%%%%%%%%%%%%%%%%%%%%
% User-defined macros
%%%%%%%%%%%%%%%%%%%%%%%%%%%%%%%%%%%%%%%%%%%%%%%%%%%%%%%%%%%%%%%%%%%%%%%%

\newcommand{\GAP}{\textsc{GAP}~\cite{GAP4}\xspace}
\newcommand{\Smallsemi}{\textsc{Smallsemi}~\cite{Smallsemi}\xspace}
\newcommand{\Semigroups}{\textsc{Semigroups}~\cite{Semigroups}\xspace}
\newcommand{\Semirings}{\textsc{Semirings}~\cite{Semirings}\xspace}
\newcommand{\bliss}{\textsc{bliss}~\cite{bliss, junttila2007}\xspace}
\newcommand{\alg}{\textsc{alg}~\cite{baueralg}\xspace}
\newcommand{\Sym}{\operatorname{Sym}}
\newcommand{\Aut}{\operatorname{Aut}}

%%%%%%%%%%%%%%%%%%%%%%%%%%%%%%%%%%%%%%%%%%%%%%%%%%%%%%%%%%%%%%%%%%%%%%%%
% Preamble ends
%%%%%%%%%%%%%%%%%%%%%%%%%%%%%%%%%%%%%%%%%%%%%%%%%%%%%%%%%%%%%%%%%%%%%%%%

\title{Counting finite semirings}
\author{J. Edwards, J. D. Mitchell, and P. Ragavan}
\date{\today}

\begin{document}

\maketitle

\begin{abstract}
  In this short note we count the finite semirings up to isomorphism
  and up to anti-isomorphism for some small values of $n$; for which we
  utilise the existing library of small semigroups in the \GAP
  package \Smallsemi.
\end{abstract}

\section{Introduction}

Enumeration of algebraic and combinatorial structures of finite order up to
isomorphism is a classical topic. Among the algebraic structures considered are
groups~\cite{BESCHE2002,A000001}, rings~\cite{Blackburn2022, Fine1993,
Kruse1970, A027623}, near-rings~\cite{Chow2024, SONATA, A305858},
semigroups and monoids~\cite{Distler2010, Distler2010, Distler2012, Distler2013,
  Forsythe1955, Grillet1996, Grillet2014, Jrgensen1977, Motzkin1955,
Plemmons1967, A027851, A058129, Satoh1994, A001426, A001423},
inverse semigroups and monoids~\cite{Malandro2019, A001428, A234843,
A234844, A234845}, and many more too numerous to mention\footnote{The disparity
  in the number of references for semigroups and monoids and the other
  algebraic structures is a consequence of the author's familiarity with the
  literature for semigroups and monoids, and there are likely many other
  references that could have been included were it not for us not knowing about
them.}. In this short note we count the number of finite semirings up to
isomorphism and up to isomorphism and anti-isomorphism for $n \leq 6$. We also
count several special classes of semirings for (slightly) larger values of $n$.

This short note was initiated by an email from M. Volkov to the second author
in February of 2025 asking if it was possible to verify the claim
in~\cite{Ren2025} that the number of ai-semirings up to isomorphism with $4$
elements is $866$. After some initial missteps it was relatively
straightforward to verify that this number is correct, by using the library of
small semigroups in the \GAP package \Smallsemi. This short note arose out from
these first steps. In contrast to groups or rings, where the numbers of
non-isomorphic objects of order $n$ is known for relatively large values of
$n$, the number of semigroups of order $10$ (up to isomorphism) is apparently
not known exactly (although the number up to isomorphism and anti-isomorphism
is known~\cite{Distler2012}). The paper \cite{Kleitman1976} purports to show
that almost all semigroups of order $n$ (as $n$ tends to infinity) up to
isomorphism and anti-isomorphism are $3$-nilpotent (and indeed $99.4\%$ of the
semigroups of order $8$ are $3$-nilpotent), and it is shown
in~\cite{Distler2012ab} that the number of $3$-nilpotent semigroups of order
$10$ is approximately $10 ^ {19}$, this number is likely close to the exact
value. Perhaps unsurprisingly, from the perspective of counting up to
isomorphism, it seems that semirings have more in common with semigroups than
with rings or groups. Very roughly speaking, rings and groups are highly
structured, providing strong constraints that facilitates their enumeration. On
the other hand, semigroups, and seemingly semirings also, are less structured,
more numerous, and harder to enumerate.

We begin with the definition of a semiring; which is a natural generalisation
of the notion of a ring.

\begin{defn}[Semiring]
  \label{def:semiring}
  A \emph{semiring} is a set \(S\) equipped with two binary
  operations \(+\) and  \(\times\) such that:
  \begin{enumerate}
    \item \((S, +)\) is a commutative semigroup ($(x + y) + z = x + (y + z)$
      and $x + y = y + x$ for all $x, y, z\in S$);
    \item \((S, \times)\) is a semigroup ($x\times (y \times z) = (x\times
      y)\times z$ for all $x,y, z\in S$); and
    \item multiplication $\times$ distributes over addition $+$
      ($x \times (y + z) = (x \times y) + (x \times z)$ and
      $(y + z) \times x = (y \times x) + (z \times x)$ for all $x, y, z \in S$).
  \end{enumerate}
\end{defn}

We note that some authors require that $(S, +)$ is a commutative monoid with
(additive) identity denoted $0$ where $x\times 0 = 0 \times x = 0$ for all
$x\in S$; see, for example, \cite{Lothaire2005, Sakarovitch2009}. We do not add
this requirement, and refer to such objects as \textit{semirings with zero}.
The numbers of semirings with zero are discussed in~\cite{stackexchange} and
in~\cite{jipsen}; see also~\alg which we will discuss a little
further below. In~\cite{Zhao2020} it was shown that there are $61$ ai-semirings
of order $3$.

Recall that if $(S, \times)$ and $(T, \otimes)$ are semigroups, then $\phi:S
\to T$ is a \textit{(semigroup) homomorphism} if $(x\times y)\phi =
(x)\phi\otimes (y)\phi$ for all $x, y\in S$. Note that we write mappings to the
right of their arguments and compose them from left to right. If $\phi: S \to
T$ is a semigroup homomorphism and $\phi$ is bijective, then $\phi$ is an
\textit{(semigroup) isomorphism}. A semigroup isomorphism from a semigroup
$(S, \times)$ to itself is called an \textit{automorphism}
and the group of all such automorphisms is denoted $\Aut(S, \times)$.

In this short note we are concerned with counting semirings up to isomorphism,
and so our next definition is that of an isomorphism.

\begin{defn}[Semiring isomorphism]\label{defn-semiring-iso}
  We say that two semirings \((S,+,\times)\) and \((S, \oplus, \otimes)\) are
  \emph{isomorphic} if there exists a bijection \(\phi: S \to S\) which is
  simultaneously a semigroup isomorphism from $(S, +)$ to $(S, \oplus)$ and
  from $(S, \times)$ to $(S, \otimes)$. We refer to $\phi$ as a
  \textit{(semiring) isomorphism}.
\end{defn}

Since a semiring is comprised of two semigroups, enumerating semirings is
equivalent to enumerating those pairs consisting of an additive semigroup $(S,
+)$ and a multiplicative semigroup $(S, \times)$ such that $\times$ distributes
over $+$. The next theorem indicates which $(S, \times)$ we should consider
for each of the additive semigroups $(S, +)$.

We denote the symmetric group on the set $S$ by $\Sym(S)$. If $\sigma\in
\Sym(S)$ and $\cdot: S \times S \to S$ is a binary operation, then we define
the binary operation $\cdot ^ \sigma: S\times S \to S$ by
\begin{equation}\label{equation-action}
  x \cdot ^ \sigma y = ((x)\sigma^{-1} \cdot (y)\sigma ^ {-1})\sigma.
\end{equation}
It is straightforward to verify that \eqref{equation-action} is a (right, group)
action of $\Sym(S)$ on the set of all binary operations on $S$. Clearly if
$\cdot$ is associative, then so too is $\cdot ^ \sigma$ for every $\sigma \in
\Sym(S)$. The group of automorphisms $\Aut(S,\cdot)$ of a
semigroup $(S, \cdot)$ coincides with the stabiliser of the operation $\cdot$
under the action of $\Sym(S)$ defined in \eqref{equation-action}.

Recall that if $H$ and $K$ are subgroups of a group $G$, then the
\textit{double cosets} $H\backslash G / K$ are the sets of the form
$\{hgk\mid h\in H, k \in K\}$ for $g\in G$.
The next theorem is key to our approach for counting semirings.

\begin{theorem}
  \label{thm:isomorphism-condition}
  Let \((S, +)\) be a commutative semigroup, let \((S, \times)\) be a semigroup,
  and let \(\sigma, \tau\in\Sym(S)\) be such that $(S, +, \times ^ \sigma)$ and
  $(S, +, \times ^ \tau)$ are semirings.
  Then \((S, +, \times ^ \sigma)\) and \((S, +, \times^\tau)\) are isomorphic
  if and only if \(\sigma\) and \(\tau\) belong to the same double coset of
  \(\Aut(S, \times) \backslash \Sym(S) / \Aut(S, +)\).
\end{theorem}

\begin{proof}
  ($\Rightarrow$) Suppose that $\phi$ is a semiring isomorphism from
  $(S, +, \times ^ \sigma)$ to $(S, +, \times^\tau)$. Then
  $\phi \in \Aut(S, +)$ and $(x \times ^ \sigma y)\phi = (x)\phi\times ^ \tau
  (y)\phi$ for all $x, y\in S$. It follows that
  \[
    ((x)\sigma ^ {-1} \times (y)\sigma ^ {-1})\sigma \phi \tau ^ {-1} =
    (x \times ^ \sigma y)\phi\tau ^ {-1}
    = ((x)\phi \times ^ {\tau} (y)\phi)\tau ^ {-1}
    = (x)\phi\tau ^ {-1} \times (y)\phi\tau ^ {-1}.
  \]
  If we set $\gamma = \sigma\phi\tau^{-1}$, $p =
  (x)\sigma^{-1}$, and $q=(y)\sigma^{-1}$, then
  $(p\times q)\gamma = (p)\gamma \times (q)\gamma$
  and so $\gamma, \gamma ^ {-1} \in \Aut(S, \times)$.
  Rearranging we obtain $\tau = \gamma^{-1}\sigma\phi$. Since $\phi \in
  \Aut(S, +)$, we conclude that $\tau$ and $\sigma$ lie in the
  same double coset of $\Aut(S, \times) \backslash \Sym(S) / \Aut(S, +)$.

  ($\Leftarrow$) Suppose that $\sigma$ and $\tau$ are in the
  same double coset of $\Aut(S, +) \backslash \Sym(S) / \Aut(S, \times)$.
  Then there exists $\alpha\in\Aut(S, \times)$ and $\beta\in\Aut(S, +)$ such
  that $\tau=\alpha\sigma\beta$.

  We will show that $\beta$ is a semiring isomorphism from $(S, +, \times
  ^{\sigma})$ to $(S, +, \times^{\tau})$. Since $\beta\in \Aut(S, +)$, it
  suffices to show that $\beta$ is an isomorphism from $(S, \times
  ^{\sigma})$ and $(S, \times^{\tau})$:
  \begin{align*}
    (x\times^\sigma y)\beta
    &= (x\sigma ^{-1} \times y\sigma ^{-1})\sigma \beta
    = (x\sigma ^{-1} \times y\sigma ^{-1})\alpha^{-1}\tau
    = (x\sigma ^{-1}\alpha^{-1} \times y\sigma ^{-1}\alpha^{-1})\tau &&
    \alpha^{-1} \in \Aut(S, \times)\\
    &= (x\beta\tau^{-1} \times y\beta\tau^{-1})\tau
    = (x\beta \times^{\tau} y\beta).\qedhere
  \end{align*}
\end{proof}

If $(S, \times)$ and $(S, \otimes)$ are semigroups, then $(S, \times)$ is said
to be \textit{anti-isomorphic} to $(S, \otimes)$ if there exists a bijection
$\phi: S \to S$ such that $(x\times y)\phi = y\otimes x$ for all $x, y\in S$.
The bijection $\phi$ is referred to as an \textit{anti-isomorphism}. If the
operations $\times$ and $\otimes$ coincide, then $\phi$ is an
\textit{anti-automorphism}. Clearly the composition of two anti-automorphisms
is an automorphism, and the composition of an anti-automorphism and an
automorphism is an anti-automorphism. As such the set of all automorphisms or
anti-automorphisms forms a group under composition of functions; we denote this
group by $\Aut^*(S, \times)$.

Similarly, using the obvious analogue of \cref{defn-semiring-iso}, we can
define anti-isomorphic semirings. It is routine to adapt the proof of
\cref{thm:isomorphism-condition} to prove the following.

\begin{cor}
  \label{cor:equiv-condition}
  Let \((S, +)\) be a commutative semigroup, let \((S, \times)\) be a
  semigroup, and let \(\sigma, \tau\in\Sym(S)\) be such that $(S, +, \times ^
  \sigma)$ and $(S, +, \times ^ \tau)$ are semirings. Then \((S, +, \times ^
  \sigma)\) and \((S, +, \times^\tau)\) are isomorphic or anti-isomorphic if
  and only if \(\sigma\) and \(\tau\) belong to the same double coset of
  \(\Aut^*(S, \times) \backslash \Sym(S) / \Aut(S, +)\).
\end{cor}

%%%%%%%%%%%%%%%%%%%%%%%%%%%%%%%%%%%%%%%%%%%%%%%%%%%%%%%%%%%%%%%%%%%%%%%%
\section{Tables of results}
%%%%%%%%%%%%%%%%%%%%%%%%%%%%%%%%%%%%%%%%%%%%%%%%%%%%%%%%%%%%%%%%%%%%%%%%

It follows from \cref{thm:isomorphism-condition} and \cref{cor:equiv-condition}
that to enumerate the semirings on a set $S$ up to isomorphism (or up to
isomorphism and anti-isomorphism) it suffices to consider every commutative
semigroup $(S, +)$ and semigroup $(S, \times)$ and to compute representatives
of the double cosets \(\Aut(S, \times) \backslash \Sym(S) / \Aut(S, +)\) (or
\(\Aut^*(S, \times) \backslash \Sym(S) / \Aut(S, +)\)). The semigroups up to
isomorphism and anti-isomorphism are available in \Smallsemi.  Since $S$ is
small, it is relatively straightforward to can compute $\Aut(S, \times)$ and
$\Aut(S, +)$ (using the \Semigroups package for \GAP, which reduces the problem
  to that of computing the automorphism group, using \bliss, of a graph
associated to $\Aut(S, \times)$). \GAP contains functionality for computing
double coset representatives based.
% TODO add "based on \cite{}."
This is the approach implemented by the authors of the current paper in the
\GAP package \Semirings to compute the numbers in this section.

Below are some tables listing the numbers of semirings (\cref{def:semiring}) up
to isomorphism, and up to isomorphism or anti-isomorphism, with various
properties for some small values of $n\in \mathbb{N}$. As far as we know, many
of the numbers in these tables were not previously known. In particular, we are
not aware of any results in the literature about the number of semirings up to
isomorphism and anti-isomorphism. Some of the numbers in the tables below can
be found using \alg, although this approach is considerably slower
than the approach described here, largely because the precomputed data for
small semigroups available in \Smallsemi does a lot of the heavy lifting.

As a sanity check, where possible, we have checked the numbers produced by
\Semirings against those in the literature, and those we could produce using
\alg. With one exception, these numbers agreed with each other. \alg computes
that there are $57,443$ semirings (using \cref{def:semiring}) of order $5$ up
to isomorphism, whereas \Semirings computes $57,427$. Using \GAP and \Semirings
we attempted to verify whether or not the output of \alg was correct.
Unfortunately, it appears that $16$ of the pairs of multiplication tables
output by \alg do not satisfy one or the other of the distributivity conditions
from \cref{def:semiring}(iii); see \cite{bauer_alg_issue16} for more details.
This precisely accounts for the difference between the number output by
\Semirings and the number output by \alg.

These tables are only a sample of the results that can be obtained
using \Semirings.

\begin{table}[ht]
  \centering
  \begin{tabular}{l|r|r}
    \multirow{2}{*}{$n$} & \multicolumn{1}{|c|}{\multirow{2}{*}{up to
    isomorphism}} & \multicolumn{1}{c}{up to isomorphism} \\
    & & \multicolumn{1}{l}{or anti-isomorphism}\\
    \midrule
    1 & 1                    & 1         \\
    2 & 10                   & 9         \\
    3 & 132                  & 106       \\
    4 & 2,341                & 1,713     \\
    5 & 57,427     & 38,247    \\
    6 & 7,571,579  & 4,102,358
  \end{tabular}
  \caption{Numbers of semirings (\cref{def:semiring}) with $n$
    elements up to isomorphism and up
    to isomorphism or anti-isomorphism. See \cite{MSsemirings} for
  \(n\leq4\) up to isomorphism.}
  \label{tab:semirings}
\end{table}

\begin{table}[ht]
  \centering
  \begin{tabular}{l|r|r|r|r|r|r|r|r}
    $n$
    & \multicolumn{4}{c|}{up to isomorphism}
    & \multicolumn{4}{c}{up to isomorphism or anti-isomorphism} \\
    \midrule
    & \multicolumn{1}{c|}{\shortstack{no additional\\constraints}} &
    \multicolumn{1}{c|}{with $0$}
    & \multicolumn{1}{c|}{with $1$} & \multicolumn{1}{c|}{with $0$ + $1$}
    & \multicolumn{1}{c|}{\shortstack{no additional\\constraints}} &
    \multicolumn{1}{c|}{with $0$}
    & \multicolumn{1}{c|}{with $1$} & \multicolumn{1}{c}{with $0$ + $1$}
    \\
    \midrule
    1 &              &          &&&          &     &  \\
    2 &              &          &&&          &     &  \\
    3 &             &         &&&          &       &\\
    4 &            &        &&&         & &\\
    5 &         &      &&&        & &\\
    6 &        &    &&&      & &\\
    7 &      &  &&&     & &\\
    8 & & &&&    & &\\
  \end{tabular}
  \caption{Numbers of commutative semirings ($x\times y = y \times
  x$ for all $x,y\in S$) with $n$ elements.}
  \label{tab:comm-semirings}
\end{table}

\begin{table}[ht]
  \centering
  \begin{tabular}{l|r|r|r|r|r|r|r|r}
    $n$
    & \multicolumn{4}{c|}{up to isomorphism}
    & \multicolumn{4}{c}{up to isomorphism or anti-isomorphism} \\
    \midrule
    & \multicolumn{1}{c|}{\shortstack{no additional\\constraints}} &
    \multicolumn{1}{c|}{with $0$}
    & \multicolumn{1}{c|}{with $1$} & \multicolumn{1}{c|}{with $0$ + $1$}
    & \multicolumn{1}{c|}{\shortstack{no additional\\constraints}} &
    \multicolumn{1}{c|}{with $0$}
    & \multicolumn{1}{c|}{with $1$} & \multicolumn{1}{c}{with $0$ + $1$}
    \\
    \midrule
    1 & 1             & &&1       & 1          & && 1      \\
    2 & 6             & &&1       & 5          & && 1      \\
    3 & 61            & &&3       & 45         & && 3      \\
    4 & 866           & &&20      & 581        & && 18     \\
    5 & 15,751        & &&149     & 9,750      & && 125    \\
    6 & 354,409       & &&1,488   & 205,744    & && 1,150  \\
    7 & 9,908,909     & &&18,554  & 5,470,437  & && 13,171 \\
    8 &               &           &&295,292 &  & && 116,274
  \end{tabular}
  \caption{Numbers of ai-semirings (i.e. those satisfying $x + x = x$ for all
    $x\in S$, where ``ai'' stands for ``additively idempotent'') with $n$
  elements.}\label{tab:ai-semirings}
\end{table}

\begin{table}[ht]
  \centering
  \begin{tabular}{l|r|r|r|r|r|r|r|r}
    $n$
    & \multicolumn{4}{c|}{up to isomorphism}
    & \multicolumn{4}{c}{up to isomorphism or anti-isomorphism} \\
    \midrule
    & \multicolumn{1}{c|}{\shortstack{no additional\\constraints}} &
    \multicolumn{1}{c|}{with $0$}
    & \multicolumn{1}{c|}{with $1$} & \multicolumn{1}{c|}{with $0$ + $1$}
    & \multicolumn{1}{c|}{\shortstack{no additional\\constraints}} &
    \multicolumn{1}{c|}{with $0$}
    & \multicolumn{1}{c|}{with $1$} & \multicolumn{1}{c}{with $0$ + $1$}
    \\
    \midrule
    1 &              &          &&&          &      & \\
    2 &              &          &&&          &      & \\
    3 &             &         &&&          &      & \\
    4 &            &        &&&         & &\\
    5 &         &      &&&        & &\\
    6 &        &    &&&      & &\\
    7 &      &  &&&     & &\\
    8 & & &&&    & &\\
  \end{tabular}
  \caption{Numbers of commutative ai-semirings ($x\times y = y \times
    x$ and $x + x = x$ for all $x,y\in S$) with $n$
  elements.}
  \label{tab:ai-semirings}
\end{table}

\begin{table}[ht]
  \centering
  \begin{tabular}{l|r|r}
    \multirow{2}{*}{$n$} & \multicolumn{1}{|c|}{\multirow{2}{*}{up to
    isomorphism}} & \multicolumn{1}{c}{up to isomorphism} \\
    & & \multicolumn{1}{l}{or anti-isomorphism}\\
    \midrule
    1 & 1         & 1      \\
    2 & 2         & 2      \\
    3 & 6         & 6      \\
    4 & 40        & 38     \\
    5 & 295       & 262    \\
    6 & 3,246     & 2,681  \\
    7 & 59,314    & 43,331
  \end{tabular}
  \caption{Numbers of semirings with one and zero with $n$ elements
    up to isomorphism or anti-isomorphism. See
  \cite{MSsemiringsWithOneAndZero} for \(n\leq6\) up to isomorphism.}
  \label{tab:semirings-with-one-and-zero}
\end{table}

\begin{table}[ht]
  \centering
  \begin{tabular}{l|r|r}
    \multirow{2}{*}{$n$} & \multicolumn{1}{|c|}{\multirow{2}{*}{up to
    isomorphism}} & \multicolumn{1}{c}{up to isomorphism} \\
    & & \multicolumn{1}{l}{or anti-isomorphism}\\
    \midrule
    1
    2
    3
    4
    5
    6
    7
    8
  \end{tabular}
  \caption{Numbers of ai-semirings with one and zero with $n$
    elements up
    to isomorphism or anti-isomorphism. See
  \cite{MSAiSemiringsWithOneAndZero} for \(n\leq7\) up to isomorphism.}
  \label{tab:ai-semirings-with-one-and-zero}
\end{table}

\begin{table}[h]
  \centering
  \begin{tabular}{l|r|r}
    \multirow{2}{*}{$n$} & \multicolumn{1}{|c|}{\multirow{2}{*}{up to
    isomorphism}} & \multicolumn{1}{c}{up to isomorphism} \\
    & & \multicolumn{1}{l}{or anti-isomorphism}\\
    \midrule
    1 & 1         & 1      \\
    2 & 4         & 4      \\
    3 & 22        & 21     \\
    4 & 169       & 155    \\
    5 & 1,819     & 1,561  \\
    6 & 41,104    & 30,112 \\
  \end{tabular}
  \caption{Numbers of semirings with one (unital semirings) with $n$
  elements up to isomorphism and up to isomorphism or anti-isomorphism.}
  \label{tab:unital-semirings}
\end{table}

%TODO(PR) add as many tables as possible that contain numbers that weren't
%known before.
% * semifields
% * zero sum free
% * commutative (with/without 0 and 1)
\printbibliography

\end{document}
