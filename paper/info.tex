\documentclass{article}
\usepackage[utf8]{inputenc}
\usepackage[cm]{fullpage}

\usepackage{booktabs}
\usepackage{amsmath, amssymb, amsthm, hyperref}
\usepackage{multicol}
\usepackage{tablefootnote}
\usepackage[shortlabels, inline]{enumitem}

\usepackage{hyperref}
\hypersetup{
  colorlinks=true,
  citecolor=black,
  filecolor=black,
  linkcolor=blue,
  urlcolor=red
}
\usepackage[capitalise, nameinlink]{cleveref}

\renewcommand{\labelenumi}{{\normalfont(\alph{enumi})}}
\setenumerate{noitemsep, topsep=0pt}

\theoremstyle{definition}
\newtheorem{defn}{Definition}[section]
\newtheorem{remark}{Remark}

\theoremstyle{plain}
\newtheorem{cor}[defn]{Corollary}
\newtheorem{theorem}[defn]{Theorem}

\setlength{\unitlength}{0.5mm}

%%%%%%%%%%%%%%%%%%%%%%%%%%%%%%%%%%%%%%%%%%%%%%%%%%%%%%%%%%%%%%%%%%%%%%%%
% Preamble ends
%%%%%%%%%%%%%%%%%%%%%%%%%%%%%%%%%%%%%%%%%%%%%%%%%%%%%%%%%%%%%%%%%%%%%%%%

\title{Enumerating finite semirings}
\author{J. D. Mitchell and P. Ragavan}
% JDM -> Pram: authors in almost all maths papers are listed alphabetically,
% not trying to take any of the credit I don't deserve :)
\date{\today}

\begin{document}

\maketitle

\begin{abstract}
  In this short note we count the finite semirings up to isomorphism
  and up to anti-isomorphism for some small values of $n$; for which we
  utilise the existing library of small semigroups from \cite{}.
\end{abstract}

\section{Introduction}

Enumeration of
algebra and combinatorial structures is a classical topic;
see~\cite{MilleniumProject, Smallsemi, graphs, digraphs, rings, etc}.
In this short note we count the number of finite semirings up to isomorphism
and up to isomorphism and anti-isomorphism for $n \leq 6$. We also count
several special classes of semirings for (slightly) larger values of $n$.

We begin with the definition of a semiring; this originates in
\cite{} and was intended as a generalisation of the notion of a ring.

\begin{defn}
  \label{def:semiring}
  A \emph{semiring} is a set \(S\) equipped with two binary
  operations \((+, \cdot)\) such that:
  \begin{enumerate}
    \item The \emph{additive reduct} \((S, +)\), which we denote by
      \(A\), is a commutative semigroup.
    \item The \emph{multiplicative reduct} \((S, \cdot)\), which we
      denote by \(M\), is a semigroup.
    \item Multiplication distributes over addition, i.e. for all \(a,
      b, c \in S\),
      \[
        a \cdot (b + c) = (a \cdot b) + (a \cdot c) \quad \text{and}
        \quad (b + c) \cdot a = (b \cdot a) + (c \cdot a).
      \]
  \end{enumerate}
  Thus, we may identify the semiring \(S\) with the pair \((A, M)\).
\end{defn}

Note that since both \(A\) and \(M\) have the same underlying set
\(S\), the symmetric group \(\operatorname{Sym}(S)\) acts naturally
on both structures.

\begin{defn}[Semiring isomorphism]
  We say that two semirings \(S = (A, M)\) and \(S' = (A', M')\) are
  \emph{isomorphic} if there exists a bijection \(\phi: S \to S'\)
  such that \(A \overset{\phi}\cong A'\) and \(M \overset{\phi}\cong
  M'\), where \(\overset{\phi}\cong\) denotes a semigroup isomorphism
  under \(\phi\).
\end{defn}

\begin{theorem}
  \label{thm:isomorphism-condition}
  Let \(A\) be a commutative semigroup and \(M\) be a semigroup, such
  that \(A\) and \(M\) are both defined on the same underlying set
  \(S\). For any permutation \(\sigma \in \operatorname{Sym}(S)\),
  let \(M^\sigma\) denote the semigroup obtained by permuting \(M\)
  via \(\sigma\). Suppose that for two permutations \(\sigma,
  \tau\in\operatorname{Sym}(S)\), the pairs \((A,M^\sigma)\) and
  \((A,M^\tau)\) both form semirings. Then, the following statements
  are equivalent:
  \begin{enumerate}
    \item \((A, M^\sigma)\) and \((A, M^\tau)\) are isomorphic.
    \item \(\sigma\) and \(\tau\) lie in the same double coset of
      \(\operatorname{Aut}(A) \backslash \operatorname{Sym}(S) /
      \operatorname{Aut}(M)\).
  \end{enumerate}
  \begin{proof}
    We will denote the product \(i\cdot j\) in \(M\) by \(M(i,j)\).

    For the forward direction, suppose \((A, M^\sigma) \cong (A,
    M^\tau)\). Equivalently, there exists a bijection \(\phi: (A,
    M^\sigma) \to (A, M^\tau)\) such that for all \(i, j \in S\):
    \begin{gather}
      \phi \in \operatorname{Aut}(A)\\
      \phi(M^\sigma(i,j)) = M^\tau(\phi(i),\phi(j))
    \end{gather}

    From (2), we have
    \begin{gather*}
      \phi\sigma(M(\sigma^{-1}(i), \sigma^{-1}(j))) =
      \tau(M(\tau^{-1}\phi(i), \tau^{-1}\phi(j))) \\
      \tau^{-1}\phi\sigma(M(\sigma^{-1}(i), \sigma^{-1}(j))) =
      M(\tau^{-1}\phi(i), \tau^{-1}\phi(j))
    \end{gather*}
    Now, let \(\gamma = \tau^{-1}\phi\sigma\), \(x =
    \sigma^{-1}(i)\text{ and }y=\sigma^{-1}(j)\). Then,
    \[
      \gamma(M(x, y)) = M(\gamma(x), \gamma(y))
    \]

    implying that \(\gamma \in \operatorname{Aut}(M) \text{ and in
    particular, } \gamma^{-1} \in \operatorname{Aut}(M)\).

    Rearranging the definition of \(\gamma\), we obtain \(\tau =
    \phi\sigma\gamma^{-1}\). Since \(\phi \in
    \operatorname{Aut}(A)\), we conclude that \(\tau\) and \(\sigma\)
    lie in the same double coset of \(\operatorname{Aut}(A)
    \backslash \operatorname{Sym}(S) / \operatorname{Aut}(M)\).

    \vspace{2em}
    For the reverse direction, we begin by supposing that \(\sigma\)
    and \(\tau\) are in the same double coset.

    Choose \(\alpha\in\operatorname{Aut}(M)\text{ and }
    \beta\in\operatorname{Aut}(A)\) such that
    \[\tau=\beta\sigma\alpha\]

    Note that for a function \(\phi\) to be an isomorphism from
    \((A,M^\sigma)\to(A, M^\tau)\), it must satisfy properties (1)
    and (2) detailed in the forward direction. In particular, if we
    have property (1), then as an automorphism of \(A\), \(\phi\)
    would automatically be bijective on the underlying set \(S\).

    We claim that \(\beta\) satisfies these properties. The first
    property follows trivially from the definition of \(\beta\). For
    the second property,

    \begin{align*}
      \beta(M^\sigma(i,j))&=\beta\sigma(M(\sigma^{-1}(i),\sigma^{-1}(j)))\\
      &=\tau\alpha^{-1}(M(\sigma^{-1}(i),\sigma^{-1}(j)))\\
      &\overset{*}{=}\tau(M(\alpha^{-1}\sigma^{-1}(i),\alpha^{-1}\sigma^{-1}(j)))\\
      &=\tau(M(\tau^{-1}\beta(i),\tau^{-1}\beta(j)))\\
      &=M^\tau(\beta(i),\beta(j))
    \end{align*}

    where we have used \(\alpha\in\operatorname{Aut}(M)\text{ and
    therefore }\alpha^{-1}\in\operatorname{Aut}(M)\) to justify the
    equality labelled `*'.

    Hence, we have shown that \(\beta\) satisfies (1) and (2) and is
    therefore the necessary isomorphism.
  \end{proof}
\end{theorem}

Note that the above theorem can be easily adapted to yield the
following result for equivalence of semirings.
\begin{cor}
  \label{cor:equiv-condition}
  Let \(A\) be a commutative semigroup and \(M\) be a semigroup, such
  that \(A\) and \(M\) are both defined on the same underlying set
  \(S\). For any permutation \(\sigma \in \operatorname{Sym}(S)\),
  let \(M^\sigma\) denote the semigroup obtained by permuting \(M\)
  via \(\sigma\). Suppose that for two permutations \(\sigma,
  \tau\in\operatorname{Sym}(S)\), the pairs \((A,M^\sigma)\) and
  \((A,M^\tau)\) both form semirings. Then, the following statements
  are equivalent:
  \begin{enumerate}
    \item \((A, M^\sigma)\) and \((A, M^\tau)\) are equivalent, i.e.
      isomorphic or anti-isomorphic.
    \item \(\sigma\) and \(\tau\) lie in the same double coset of
      \(\operatorname{Aut}(A) \backslash \operatorname{Sym}(S) /
      \operatorname{Aut}^*(M)\).
  \end{enumerate}
  where \(\operatorname{Aut}^*(M)\) denotes the group of
  automorphisms and anti-automorphisms of \(M\).
\end{cor}

\begin{remark}
  \(\operatorname{Aut}(M)\) and \(\operatorname{Aut}^*(M)\) are equal
  if \(M\) is non-self-dual.
\end{remark}

\begin{remark}
  Suppose \(M\) is a self-dual semigroup and \(\tau\) is an
  anti-automorphism on \(M\). Then,
  \[\operatorname{Aut}^*(M) =
    \langle\operatorname{Generators}(\operatorname{Aut}(M)) \cup
  \{\tau\}\rangle.\]
\end{remark}
\vspace{3em}
\begin{center}
  \textbf{Results}
\end{center}
The following results were obtained using the \texttt{aisemirings}
package in GAP, which (for small \(n\)) provides functions for
counting and enumerating various semiring-related objects. First, we
introduce a few definitions for the less known structures.

Note that we take the convention that a ring has a multiplicative
identity, although a semiring might not.

\begin{defn}
  The prefix `ai' refers to \emph{additive idempotence}. For
  instance, an ai-semiring \(S\) is a semiring such that the additive
  reduct is idempotent, i.e. it satisfies \(a + a = a\) for all \(a\in S\).
\end{defn}
\begin{defn}
  A \emph{rng} is a ring without the requirement for multiplicative
  identity. Maintaining the language from
  Definition~\ref{def:semiring}, this may be thought of as the pair
  \((A, M)\) where \(A\) is an abelian group and \(M\) is a semigroup.
\end{defn}

\begin{defn}
  \label{def:rig}
  A \emph{rig} \(S\) is a ring without the requirement for negatives
  (additive inverses) such that
  \begin{equation}
    \label{eq:rig-axiom}
    0\cdot a=0\qquad\forall a\in S,
  \end{equation}
  where \(0\) denotes the additive identity in \(S\).

  Note that although generally in a ring,
  Property~\eqref{eq:rig-axiom} follows directly from the axioms,
  this might not hold if we do not have negatives, and so is instead
  specified explicitly as an axiom.

  As above, this may be thought of as the pair \((A, M)\) where \(A\)
  is a commutative monoid and \(M\) is a monoid. In fact, many
  authors use the term `rig' to refer to what we have defined as a semiring.

\end{defn}
\begin{defn}
  A \emph{rg} \(S\) is a ring without the requirement for negatives
  or multiplicative identity, such that
  \begin{equation*}
    0\cdot a=0\qquad\forall a\in S,
  \end{equation*}
  where \(0\) denotes the additive identity in \(S\).

  The reasoning for this additional axiom is as in
  Definition~\ref{def:rig}. As above, this may be thought of as the
  pair \((A, M)\) where \(A\) is a commutative monoid and \(M\) is a
  semigroup. The term `rg' is very non-standard in the literature.
\end{defn}

The semiring-like structures that can be counted using the
\texttt{aisemirings} package are:

\begin{multicols}{2}
  \begin{itemize}
    \item Semirings
    \item Semirings with one
    \item Ai-semirings
    \item Ai-semirings with one
    \item Rigs
    \item Ai-rigs
    \item Rgs
    \item Ai-rgs
    \item Rngs
    \item Rings
  \end{itemize}
\end{multicols}

For instance, one could count the number of semirings with \(n\)
elements up to isomorphism using \texttt{NrSemirings(n)} or up to
equivalence using \texttt{NrSemirings(n, true)}.
\texttt{AllSemirings} could be used to enumerate these objects.
Functions for the other objects mentioned above are constructed
similarly. Using the helper function \texttt{SETUPFINDER}, the
package could also easily be used to count/enumerate any object which
is a semiring with additional constraints, as long as the sets of
valid additive and multiplicative reducts are expressible as
\href{https://gap-packages.github.io/smallsemi/doc/chap4_mj.html#X82F9C36C86006857}{families
of semigroups}. With a little more effort, it would be possible to
count semirings for arbitrary sets of valid additive and multiplicative reducts.

The algorithm used to count these objects up to isomorphism is fairly
rudimentary and is based on Theorem~\ref{thm:isomorphism-condition}.
As the condition given by this theorem is precise, we can make use of
the \texttt{smallsemi} package in GAP to loop over possible semirings
\((A, M)\) in a minimal way, such that no two semirings yielded by
this process can be isomorphic.

Similarly, we can count semirings up to equivalence by using the
condition given in Corollary~\ref{cor:equiv-condition}, again in a minimal way.

Below are some tables of results for the aforementioned structures.
As far as we know, no results are published the number of any of
these structures up to equivalence. For results up to isomorphism,
those that have not been previously published (as far as we know) are
marked `$^\dagger$'. Results that we are in the process of computing
are marked `?'. As a sanity check, various results that are already
published are available at Peter Jipsen's
\href{https://math.chapman.edu/~jipsen/structures/doku.php?id=start}{Mathematical
Structures Library}, though he makes use of different naming
conventions\footnote{Note that Jipsen's
  \href{https://math.chapman.edu/~jipsen/structures/doku.php?id=semirings_with_identity\#finite_members}{page
  for ``semirings with one''}, seems to be mistitled and actually
  provides results for ai-semirings with one (which can be counted
  using the \texttt{ai-semirings} package!). This is not a difference
  in naming convention, but seems to just be a mistake. As far as we
know, all results in Table~\ref{tab:unital-semirings} are unpublished.}.
\begin{table}[ht]
  \centering
  \begin{tabular}{l|r|r}
    \toprule
    $n$ & up to isomorphism & up to isomorphism + anti-isomorphism \\
    \midrule
    1 & 1         & 1         \\
    2 & 10        & 9         \\
    3 & 132       & 106       \\
    4 & 2,341     & 1,713     \\
    5 & 57,427$^\dagger$    & 38,247    \\
    6 & 7,571,579$^\dagger$  & 4,102,358 \\
  \end{tabular}
  \caption{Numbers of semirings with $n$ elements up to isomorphism and up
    to isomorphism or anti-isomorphism. See
    \href{https://math.chapman.edu/~jipsen/structures/doku.php?id=semirings\#finite_members}{Jipsen's
  library} for \(n\leq4\) up to isomorphism.}
  \label{tab:semirings}
\end{table}

\begin{table}[ht]
  \centering
  \begin{tabular}{l|r|r}
    \toprule
    $n$ & up to isomorphism & up to isomorphism + anti-isomorphism \\
    \midrule
    1 & 1         & 1       \\
    2 & 6         & 5       \\
    3 & 61        & 45      \\
    4 & 866       & 581     \\
    5 & 15,751$^\dagger$    & 9,750   \\
    6 & 354,409$^\dagger$   & 205,744 \\
    7 & 9,908,909$^\dagger$ & 5,470,437       \\
  \end{tabular}
  \caption{Numbers of ai-semirings with $n$ elements up to isomorphism and up
    to isomorphism or anti-isomorphism. See
    \href{https://math.chapman.edu/~jipsen/structures/doku.php?id=idempotent_semirings\#finite_members}{Jipsen's
  library} for \(n\leq4\) up to isomorphism.}
  \label{tab:ai-semirings}
\end{table}

\begin{table}[ht]
  \centering
  \begin{tabular}{l|r|r}
    \toprule
    $n$ & up to isomorphism & up to isomorphism + anti-isomorphism \\
    \midrule
    1 & 1         & 1      \\
    2 & 2         & 2      \\
    3 & 6         & 6      \\
    4 & 40        & 38     \\
    5 & 295       & 262    \\
    6 & 3,246     & 2,681  \\
    7 & 59,314$^\dagger$    & 43,331 \\
  \end{tabular}
  \caption{Numbers of rigs with $n$ elements up to isomorphism and up
    to isomorphism or anti-isomorphism. See
    \href{https://math.chapman.edu/~jipsen/structures/doku.php?id=semirings_with_identity_and_zero\#finite_members}{Jipsen's
  library} for \(n\leq6\) up to isomorphism.}
  \label{tab:rigs}
\end{table}

\begin{table}[ht]
  \centering
  \begin{tabular}{l|r|r}
    \toprule
    $n$ & up to isomorphism & up to isomorphism + anti-isomorphism \\
    \midrule
    1 & 1         & 1      \\
    2 & 1         & 1      \\
    3 & 3         & 3      \\
    4 & 20        & 18     \\
    5 & 149       & 125    \\
    6 & 1,488     & 1,150  \\
    7 & 18,554    & 13,171 \\
    8 & 295,292$^\dagger$   & 116,274      \\
  \end{tabular}
  \caption{Numbers of ai-rigs with $n$ elements up to isomorphism and up
    to isomorphism or anti-isomorphism. See
    \href{https://math.chapman.edu/~jipsen/structures/doku.php?id=idempotent_semirings_with_identity_and_zero\#finite_members}{Jipsen's
  library} for \(n\leq7\) up to isomorphism.}
  \label{tab:ai-rigs}
\end{table}

\begin{table}[h]
  \centering
  \begin{tabular}{l|r|r}
    \toprule
    $n$ & up to isomorphism & up to isomorphism + anti-isomorphism \\
    \midrule
    1 & 1         & 1      \\
    2 & 4         & 4      \\
    3 & 22        & 21     \\
    4 & 169       & 155    \\
    5 & 1,819     & 1,561  \\
    6 & 41,104    & 30,112 \\
    7 & ?         & ?      \\
  \end{tabular}
  \caption{Numbers of semirings with one (unital semirings) with $n$
  elements up to isomorphism and up to isomorphism or anti-isomorphism.}
  \label{tab:unital-semirings}
\end{table}

These tables are merely a sample of the results that can be obtained
using the \texttt{aisemirings} package.

\end{document}
