\documentclass{article}
\usepackage{booktabs} % For professional tables
\usepackage{amsmath, amssymb, amsthm} % Recommended for math symbols
\setlength{\parskip}{5pt}  
\setlength{\parindent}{0pt}
\newtheorem{theorem}{Theorem}

\begin{document}
\begin{theorem}
    Let \((A, +)\) be an abelian group and \((M, \cdot)\) a group such that \((A, M)\) forms a semiring with the usual operations. For any permutation \(\sigma \in \operatorname{Sym}(M)\), let \(M^\sigma\) denote the semigroup obtained by permuting \(M\) via \(\sigma\). Then the following statements are equivalent:
    \begin{enumerate}
        \item The semirings \((A, M^\sigma)\) and \((A, M^\tau)\) are isomorphic.
        \item \(\sigma\) and \(\tau\) lie in the same double coset of \(\operatorname{Aut}(A) \backslash \operatorname{Sym}(M) / \operatorname{Aut}(M)\).
    \end{enumerate}
\begin{proof}
    We will denote the product \(i\cdot j\) in \(M\) by \(M(i,j)\).

    For the forward direction, suppose \((A, M^\sigma) \cong (A, M^\tau)\). Equivalently, there exists an isomorphism \(\phi: (A, M^\sigma) \to (A, M^\tau)\) such that:
    \begin{gather}
         \phi \in \operatorname{Aut}(A)\\
         \phi(M^\sigma(i,j)) = M^\tau(\phi(i),\phi(j))
    \end{gather}

    Writing \(\gamma = \tau^{-1}\phi\sigma\), \(x = \sigma^{-1}(i)\text{ and } y = \sigma^{-1}(j)\) where necessary, we have
    \begin{align*}
        (2) &\implies \phi\sigma(M(\sigma^{-1}(i), \sigma^{-1}(j))) = \tau(M(\tau^{-1}\phi(i), \tau^{-1}\phi(j))) \\
        &\implies \tau^{-1}\phi\sigma(M(\sigma^{-1}(i),\sigma^{-1}(j))) = M(\tau^{-1}\phi(i), \tau^{-1}\phi(j))\\
        &\implies \gamma(M(\sigma^{-1}(i), \sigma^{-1}(j)) = M(\gamma\sigma^{-1}(i), \gamma\sigma^{-1}(j))  \\
        &\implies \gamma M(x,y)=M(\gamma(x), \gamma(y))\\
        &\implies \gamma\in\operatorname{Aut}(M)\\
        &\implies \gamma^{-1}\in\operatorname{Aut}(M)\\
    \end{align*}
    Now, by rearranging to \(\tau = \phi\sigma\gamma^{-1}\), and noting that from (1) we have that \(\phi\in\operatorname{Aut}(A)\), we have shown that \(\tau\) and \(\sigma\) are in the same double coset of \(\operatorname{Aut}(A)\backslash\operatorname{Sym(M)}/\operatorname{Aut}(M)\)

    For the reverse direction, we begin by supposing that \(\sigma\) and \(\tau\) are in the same double coset.

    Choose \(\alpha\in\operatorname{Aut}(M)\text{ and } \beta\in\operatorname{Aut}(A)\) such that 
    \[\tau=\beta\sigma\alpha\]

    Note that for a function \(\phi\) to be an isomorphism from \((A,M^\sigma)\to(A, M^\tau)\), it must satisfy properties (1) and (2) detailed in the forward direction.

    We claim that \(\beta\) satisfies these properties. The first property follows trivially from the definition of \(\beta\). For the second property, 

    \begin{align*}
        \beta(M^\sigma(i,j)&=\beta\sigma(M(\sigma^{-1}(i),\sigma^{-1}(j)))\\
        &=\tau\alpha^{-1}M(\sigma^{-1}(i),\sigma^{-1}(j))\\
        &\overset{*}{=}\tau M(\alpha^{-1}\sigma^{-1}(i),\alpha^{-1}\sigma^{-1}(j))\\
        &=\tau M(\tau^{-1}\beta(i),\tau^{-1}\beta(j))\\
        &=M^\tau(\beta(i),\beta(j))
    \end{align*}

    where we have used the fact that \(\alpha\in\operatorname{Aut}(M)\text{ and therefore }\alpha^{-1}\in\operatorname{Aut}(M)\) to justify the equality labelled `*'.

    Hence, we have shown that \(\beta\) satisfies (1) and (2) and is therefore the necessary isomorphism.
\end{proof}





  
\end{theorem}

\begin{table}[h]
  \centering
  \begin{tabular}{l|r|r}
    \toprule
    $n$ & up to isomorphism & up to isomorphism + anti-isomorphism \\
    \midrule
    1 & 1      & 1 \\
    2 & 6      & 5 \\
    3 & 61     & ? \\
    4 & 866    & ? \\
    5 & 15,751 & ? \\
    6 & 354,409 & ? \\
  \end{tabular}
  \caption{Numbers of ai-semirings with $n$ elements up to isomorphism and up
  to isomorphism or anti-isomorphism.}
  \label{tab:example}
\end{table}

\end{document}