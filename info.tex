\documentclass{article}
\usepackage{booktabs} % For professional tables
\usepackage{amsmath, amssymb, amsthm, hyperref} % Recommended for math symbols
\usepackage{multicol}
\hypersetup{
    colorlinks=true,
    linkcolor=blue,
    filecolor=magenta,      
    urlcolor=cyan,
    pdftitle={Overleaf Example},
    pdfpagemode=FullScreen,
    }
\setlength{\parskip}{5pt}  
\setlength{\parindent}{0pt}
\theoremstyle{definition}
\newtheorem{theorem}{Theorem}
\newtheorem{definition}{Definition}
\newtheorem{cor}{Corollary}
\newtheorem{remark}{Remark}

\begin{document}
\begin{center}
    \textbf{Preliminaries}
\end{center}
\begin{definition}
    A \emph{semiring} is a set \(S\) equipped with two binary operations \((+, \cdot)\) such that:
    \begin{itemize}
        \item The \emph{additive reduct} \((S, +)\), which we denote by \(A\), is a commutative semigroup.
        \item The \emph{multiplicative reduct} \((S, \cdot)\), which we denote by \(M\), is a semigroup.
        \item Multiplication distributes over addition, i.e. for all \(a, b, c \in S\),
        \[
        a \cdot (b + c) = (a \cdot b) + (a \cdot c) \quad \text{and} \quad (b + c) \cdot a = (b \cdot a) + (c \cdot a).
        \]
    \end{itemize}
    Thus, we may identify the semiring \(S\) with the pair \((A, M)\).
\end{definition}
    Note that since both \(A\) and \(M\) have the same underlying set \(S\), the symmetric group \(\operatorname{Sym}(S)\) acts naturally on both structures.
\begin{definition}[Semiring isomorphism]
    We say that two semirings \(S = (A, M)\) and \(S' = (A', M')\) are \emph{isomorphic} if there exists a bijection \(\phi: S \to S'\) such that \(A \overset{\phi}\cong A'\) and \(M \overset{\phi}\cong M'\), where \(\overset{\phi}\cong\) denotes a semigroup isomorphism under \(\phi\).
\end{definition}

\begin{theorem}
    Let \(A\) be a commutative semigroup and \(M\) be a semigroup, such that \(A\) and \(M\) are both defined on the same underlying set \(S\). For any permutation \(\sigma \in \operatorname{Sym}(S)\), let \(M^\sigma\) denote the semigroup obtained by permuting \(M\) via \(\sigma\). Suppose that for two permutations \(\sigma, \tau\in\operatorname{Sym}(S)\), the pairs \((A,M^\sigma)\) and \((A,M^\tau)\) both form semirings. Then, the following statements are equivalent:
    \begin{enumerate}
        \item \((A, M^\sigma)\) and \((A, M^\tau)\) are isomorphic.
        \item \(\sigma\) and \(\tau\) lie in the same double coset of \(\operatorname{Aut}(A) \backslash \operatorname{Sym}(S) / \operatorname{Aut}(M)\).
    \end{enumerate}
\begin{proof}
    We will denote the product \(i\cdot j\) in \(M\) by \(M(i,j)\).

    For the forward direction, suppose \((A, M^\sigma) \cong (A, M^\tau)\). Equivalently, there exists a bijection \(\phi: (A, M^\sigma) \to (A, M^\tau)\) such that for all \(i, j \in S\):
    \begin{gather}
         \phi \in \operatorname{Aut}(A)\\
         \phi(M^\sigma(i,j)) = M^\tau(\phi(i),\phi(j))
    \end{gather}

    From (2), we have
    \begin{gather*}
        \phi\sigma(M(\sigma^{-1}(i), \sigma^{-1}(j))) = \tau(M(\tau^{-1}\phi(i), \tau^{-1}\phi(j))) \\
        \tau^{-1}\phi\sigma(M(\sigma^{-1}(i), \sigma^{-1}(j))) = M(\tau^{-1}\phi(i), \tau^{-1}\phi(j))
    \end{gather*}
    Now, let \(\gamma = \tau^{-1}\phi\sigma\), \(x = \sigma^{-1}(i)\text{ and }y=\sigma^{-1}(j)\). Then,
    \[
        \gamma(M(x, y)) = M(\gamma(x), \gamma(y))
    \]

    implying that \(\gamma \in \operatorname{Aut}(M) \text{ and in particular, } \gamma^{-1} \in \operatorname{Aut}(M)\). 
    
    Rearranging the definition of \(\gamma\), we obtain \(\tau = \phi\sigma\gamma^{-1}\). Since \(\phi \in \operatorname{Aut}(A)\), we conclude that \(\tau\) and \(\sigma\) lie in the same double coset of \(\operatorname{Aut}(A) \backslash \operatorname{Sym}(S) / \operatorname{Aut}(M)\).

    \vspace{2em}
    For the reverse direction, we begin by supposing that \(\sigma\) and \(\tau\) are in the same double coset.

    Choose \(\alpha\in\operatorname{Aut}(M)\text{ and } \beta\in\operatorname{Aut}(A)\) such that 
    \[\tau=\beta\sigma\alpha\]

    Note that for a function \(\phi\) to be an isomorphism from \((A,M^\sigma)\to(A, M^\tau)\), it must satisfy properties (1) and (2) detailed in the forward direction. In particular, if we have property (1), then as an automorphism of \(A\), \(\phi\) would automatically be bijective on the underlying set \(S\).

    We claim that \(\beta\) satisfies these properties. The first property follows trivially from the definition of \(\beta\). For the second property, 

    \begin{align*}
        \beta(M^\sigma(i,j))&=\beta\sigma(M(\sigma^{-1}(i),\sigma^{-1}(j)))\\
        &=\tau\alpha^{-1}(M(\sigma^{-1}(i),\sigma^{-1}(j)))\\
        &\overset{*}{=}\tau(M(\alpha^{-1}\sigma^{-1}(i),\alpha^{-1}\sigma^{-1}(j)))\\
        &=\tau(M(\tau^{-1}\beta(i),\tau^{-1}\beta(j)))\\
        &=M^\tau(\beta(i),\beta(j))
    \end{align*}

    where we have used \(\alpha\in\operatorname{Aut}(M)\text{ and therefore }\alpha^{-1}\in\operatorname{Aut}(M)\) to justify the equality labelled `*'.

    Hence, we have shown that \(\beta\) satisfies (1) and (2) and is therefore the necessary isomorphism.
\end{proof}
\end{theorem}

Note that the above theorem can be easily adapted to yield the following result for equivalence of semirings.
\begin{cor}
    Let \(A\) be a commutative semigroup and \(M\) be a semigroup, such that \(A\) and \(M\) are both defined on the same underlying set \(S\). For any permutation \(\sigma \in \operatorname{Sym}(S)\), let \(M^\sigma\) denote the semigroup obtained by permuting \(M\) via \(\sigma\). Suppose that for two permutations \(\sigma, \tau\in\operatorname{Sym}(S)\), the pairs \((A,M^\sigma)\) and \((A,M^\tau)\) both form semirings. Then, the following statements are equivalent:
    \begin{enumerate}
        \item \((A, M^\sigma)\) and \((A, M^\tau)\) are equivalent, i.e. isomorphic or anti-isomorphic.
        \item \(\sigma\) and \(\tau\) lie in the same double coset of \(\operatorname{Aut}(A) \backslash \operatorname{Sym}(S) / \operatorname{Aut}^*(M)\).
    \end{enumerate}
    where \(\operatorname{Aut}^*(M)\) denotes the group of automorphisms and anti-automorphisms of \(M\).
\end{cor}

\begin{remark}
    \(\operatorname{Aut}(M)\) and \(\operatorname{Aut}^*(M)\) are equal if \(M\) is non-self-dual.
\end{remark}

\begin{remark}
    Suppose \(M\) is a self-dual semigroup and \(\tau\) is an anti-automorphism on \(M\). Then, 
    \[\operatorname{Aut}^*(M) = \langle\operatorname{Generators}(\operatorname{Aut}(M)) \cup \{\tau\}\rangle.\]
\end{remark}
\vspace{3em}
\begin{center}
    \textbf{Results}
\end{center}
The following results were obtained using the \texttt{semirings} package in GAP, which (for small \(n\)) provides functions for counting and enumerating the following objects\footnote{In the list of objects, the prefix `ai' refers to additive idempotence, `rng' refers to a ring without multiplicative identity, `rig' refers to a ring without negatives, `rg' refers to a ring without negatives or multiplicative identity}:
\begin{multicols}{2}
\begin{itemize}
    \item Semirings
    \item Semirings with one
    \item Ai-semirings
    \item Ai-semirings with one
    \item Rigs
    \item Ai-rigs
    \item Rgs
    \item Ai-rgs
    \item Rngs
    \item Rings
\end{itemize}
\end{multicols}

For instance, one could count the number of semirings with \(n\) elements up to isomorphism using \texttt{NrSemirings(n)} or up to equivalence using \texttt{NrSemirings(n, true)}. \texttt{AllSemirings} could be used to enumerate these objects. Functions for the other objects mentioned above are constructed similarly. Using the helper function \texttt{SETUPFINDER}, the package could also easily be used to count/enumerate any object which is a semiring with additional constraints, as long as the sets of valid additive and multiplicative reducts are expressible as \href{https://gap-packages.github.io/smallsemi/doc/chap4_mj.html#X82F9C36C86006857}{families of semigroups}. With a little more effort, it would be possible to count semirings for arbitrary sets of valid additive and multiplicative reducts.
\begin{table}[ht]
    \centering
    \begin{tabular}{l|r|r}
      \toprule
      $n$ & up to isomorphism & up to isomorphism + anti-isomorphism \\
      \midrule
      1 & 1         & 1         \\
      2 & 10        & 9         \\
      3 & 132       & 106       \\
      4 & 2,341     & 1,713     \\
      5 & 57,427    & 38,247    \\
      6 & 7,571,579 & 4,102,358 \\
    \end{tabular}
    \caption{Numbers of semirings with $n$ elements up to isomorphism and up
    to isomorphism or anti-isomorphism.}
    \label{tab:semirings}
\end{table}

\begin{table}[ht]
  \centering
  \begin{tabular}{l|r|r}
    \toprule
    $n$ & up to isomorphism & up to isomorphism + anti-isomorphism \\
    \midrule
    1 & 1         & 1       \\
    2 & 6         & 5       \\
    3 & 61        & 45      \\
    4 & 866       & 581     \\
    5 & 15,751    & 9,750   \\
    6 & 354,409   & 205,744 \\
    7 & 9,908,909 & ?       \\
  \end{tabular}
  \caption{Numbers of ai-semirings with $n$ elements up to isomorphism and up
  to isomorphism or anti-isomorphism.}
  \label{tab:ai-semirings}
\end{table}

\begin{table}[ht]
    \centering
    \begin{tabular}{l|r|r}
      \toprule
      $n$ & up to isomorphism & up to isomorphism + anti-isomorphism \\
      \midrule
      1 & 1         & 1      \\
      2 & 2         & 2      \\
      3 & 6         & 6      \\
      4 & 40        & 38     \\
      5 & 295       & 262    \\
      6 & 3,246     & 2,681  \\
      7 & 59,314    & 43,331 \\
    \end{tabular}
    \caption{Numbers of rigs with $n$ elements up to isomorphism and up
    to isomorphism or anti-isomorphism.}
    \label{tab:rigs}
\end{table}

\begin{table}[ht]
    \centering
    \begin{tabular}{l|r|r}
      \toprule
      $n$ & up to isomorphism & up to isomorphism + anti-isomorphism \\
      \midrule
      1 & 1         & 1      \\
      2 & 1         & 1      \\
      3 & 3         & 3      \\
      4 & 20        & 18     \\
      5 & 149       & 125     \\
      6 & 1,488     & 1,150   \\
      7 & 18,554    & 13,171   \\
      8 & ?         & ?        \\
    \end{tabular}
    \caption{Numbers of ai-rigs with $n$ elements up to isomorphism and up
    to isomorphism or anti-isomorphism.}
    \label{tab:ai-rigs}
\end{table}

\begin{table}[ht]
    \centering
    \begin{tabular}{l|r|r}
      \toprule
      $n$ & up to isomorphism & up to isomorphism + anti-isomorphism \\
      \midrule
      1 & 1         & 1 \\
      2 & 4         & 4 \\
      3 & 22        & 21 \\
      4 & 169       & 155 \\
      5 & 1,819     & 1,561 \\
      6 & 41,104    & 30,112 \\
      7 & ?         & ? \\
    \end{tabular}
    \caption{Numbers of unital semirings with $n$ elements up to isomorphism and up
    to isomorphism or anti-isomorphism.}
    \label{tab:unital-semirings}
\end{table}

\end{document}